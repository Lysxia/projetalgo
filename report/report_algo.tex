\documentclass[a4paper]{article}

\usepackage[french]{babel}
\usepackage[T1]{fontenc}
\usepackage[utf8]{inputenc}
\usepackage{fullpage}
\usepackage{pgf}
\usepackage{tikz}

\begin{document}

\title{Projet de programmation et Algorithmique}
\author{Antoine Voizard, Li-yao Xia}
\date{\today}

\maketitle
%\normalsize

\section{Composition du projet}

Initialement écrit en C++ pour les templates, le projet est maintenant
écrit en C.

\subsection{Fichiers source}

\begin{itemize}
  \item {\it randpi.c .h} : générateur de matrices pseudo-aléatoires ;
  \item {\it matrixio.c .h} : lecture et affichage de matrices ;
  \item {\it naïve\_mult.c .h} : implémentation de la multiplication naïve
  de matrices en $O(m \cdot n \cdot p)$, avec parenthésage naïf ;
  \item {\it extracted.c .h} : fonctions auxiliaires utilisées par
  la fonction {\bf strassen} ;
  \item {\it strassen.c .h} : implémentation(s) de l'algorithme de
  Strassen ; (avec éventuellement une optimisation)
  \item {\it parenthesization.c .h} : implémentation de l'algorithme de
  parenthésage optimal par programmation dynamique.
  \item {\it multiply.c .h} : multiplication rapide d'une suite de
  matrices en utilisant les deux algorithmes.
  \item {\it m2m.c .h} : programme. Multiplie deux matrices avec
  l'algorithme de Strassen.
  \item {\it naïve.c .h} : programme. Multiplie une suite de matrices
  de manière naïve.
  \item {\it longproduct.c .h} : programme. Multiplie une suite de
  matrices avec algorithme de Strassen pour un parenthésage optimal.
  \item {\it demo.c .h} : programme de démonstration. Multiplie des
  matrices générées pseudo-aléatoirement et compare les approches
  optimisée et naïve.
\end{itemize}

\subsection{Fichiers auxiliaires}

\begin{itemize}
  \item {\it Makefile} ;
  \item {\it randmatrices.c} : générateur de matrices pseudo-aléatoires,
  graine laissée par défaut car l'algorithme de Strassen n'est pas
  influencé par les valeurs des coefficients ;

\subsection{Usage, cibles de la commande make}

Certains exécutables sont produits en deux versions : la version 1
arrête la récursion de l'algorithme de Strassen
quand une dimension vaut 1, la version 2 arrête quand
une dimension arrive à moins de 50.


\begin{description}
  \item[default (make)] génère tous les exécutables associés au projet.


  
\end{itemize}

\section{Algorithme de Strassen}

Ce qui suit sera présenté le mardi 18 décembre 2012.

L'algorithme a été implémenté sans difficultés, en prenant quelques
libertés sur certains points.

\subsection{Représentation des données}

Les matrices sont représentées de manière la plus compacte possible :
une matrice $A=(a_{i,j})_{0 \leq i < m, 0 \leq j < n}$ $m \times n$ est
un tableau {\bf A} d'{\bf int} de taille $m \cdot n$,
la case $a_{i,j}$ correspondant à la case {\bf A[i*n+j]}.

Les tailles des matrices sont de type {\bf unsigned int} ({\bf unsigned}).

Pour éviter de recopier les données, on définit une représentation en
sous-matrices :
une sous-matrice
$M = (m_{i',j'})_{0 \leq i' < m_0, 0 \leq j' < n_0} =
(a_{i_0+i',j_0+j'})_{i',j'}$
de $A$ est représentée par la donnée de :

\begin{itemize}
  \item le pointeur {\bf A+i0*n+j0} ;
  \item les tailles {\bf m0}, {\bf n0} ;
  \item la largeur de {\bf A}, {\bf n}.
\end{itemize}

\subsection{Détails d'implémentation}

La fonction

\begin{verbatim}
 int* strassen(int* A, int* B, int m, int n, int p);
\end{verbatim}

prend en argument deux matrices {\bf A}, {\bf B} de tailles {\bf m*n} et
{\bf n*p} et renvoie un pointeur sur une matrice de taille {\bf m*p}.

Elle fait appel à une (sous-)fonction

\begin{verbatim}
  int* _strassen(int* A, int* B, int* C,
                 int m, int n, int p,
                 int width_A, int width_B, int width_C);
\end{verbatim}

prend en argument trois {\it sous-matrices} et calcule le produit
$A \cdot B$ dans {\bf C} qui devra être déjà allouée avant d'appeler
{\bf \_strassen}.

Une fonction {\bf \_add} avec des conventions d'appel similaires
est aussi utilisée.

Dans la première version, on arrête l'algorithme quand une dimension
vaut 1.

On alloue trois matrices de tailles {\bf M}:$m \cdot n$,
{\bf N}:$n \cdot p$, {\bf X}:$m \cdot p$
pour contenir les résultats intermédiaires.

Les produits $X4$, $X5$, $X6$, $X7$ sont calculés directement dans les
sous-matrices $P21$, $P12$, $P22$, $P11$, ce qui est permis grâce
à la convention utilisée pour {\bf \_strassen}.

Dans le cas de dimension(s) impaires, on n'ajoute pas de ligne/colonne
mais on a trouvé une formule générale pour calculer les 7 produits.
(à expliciter dans la présentation)

\subsection{Tentatives d'optimisation}

Nous avons essayé d'utiliser l'arithmétique des pointeurs dans la
multiplication naïve qui termine la récursion de {\bf \_strassen}
mais cela ralentit le programme.

L'autre optimisation, qui est concluante, est d'arrêter la récursion
plus tôt : on utilise la multiplication naïve 
quand la plus petite dimension est inférieure à une taille
$s$. L'optimum trouvé (par expérimentation) est environ $s=50$

Cette modification a divisé le temps de calcul
par un facteur entre $5$ et $10$.

\section{Parenthésage optimal}

\end{document}

