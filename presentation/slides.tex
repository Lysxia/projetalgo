\documentclass{beamer}
\usepackage[french]{babel}
\usepackage[utf8]{inputenc}
\usepackage{pgfplots}
\usepackage{qtree}
%\usetheme{Warsaw}
\title[Projet de programmation et d'algorithmique]{
Algorithme de Strassen et parenthésage optimal
pour la mutliplication de matrices}
\author{Antoine Voizard, Li-yao Xia}
\date{18 décembre 2012}


\begin{document}

\begin{frame}
\titlepage
\end{frame}

\AtBeginSection[]
{
  \begin{frame}<beamer>
    \frametitle{\insertsection}

    \tableofcontents[currentsection]
  \end{frame}
}

\begin{frame}{Introduction}
  Objectifs :
  \begin{itemize}
    \item Ecrire du C
    \item Faire mieux que les algorithmes naifs
  \end{itemize}
\end{frame}

\section{Algorithme de Strassen}

\subsection{Principe}
\begin{frame}[fragile]
  \frametitle{\insertsubsection}
  \begin{itemize}
    \item Multiplication rapide de matrices sur un anneau
    (ici, des \verb=int=)
    \item Limiter le nombre de multiplications ; en taille $2 \times 2$ :
    \begin{itemize}
      \item mult. naïve : 8 multiplications scalaires
      \item mult. de Strassen : 7 multiplications
    \end{itemize}
    \item Calcul par blocs, algorithme récursif
  \end{itemize}
\end{frame}

\subsection{Implémentation et optimisation}
\begin{frame}[fragile]
  \frametitle{\insertsubsection}
  \framesubtitle{Structures de données}
  Représentation compacte des données
  \begin{itemize}
    \item<2-> Matrice $A$ de taille $m\times n$
    \begin{itemize}
      \item Tableau \verb=A= d'\verb=int=
      (pointeur sur la première case) de taille \verb=m*n=
      %alloué dynamiquement
      \item Accès à la case $(i,j)$ : \verb=A[i*n+j]=
    \end{itemize}
    \item<3-> Sous-matrice $M$ extraite de $A$ de taille $m'\times n'$
    \begin{itemize}
      \item Pointeur sur la première case de $M$ dans \verb=A=
      \item Largeur de la matrice $A$, \verb=n=
      \item Accès à la case $(i,j)$ : \verb=M[i*n+j]=
    \end{itemize}
  \end{itemize}
\end{frame}

\begin{frame}[fragile]
  \frametitle{\insertsubsection}
  \framesubtitle{Implémentation}
  \small\begin{verbatim}
  int* strassen(int* A, int* B, int m, int n, int p);
  \end{verbatim}\normalsize

  \begin{itemize}
    \item Opérations auxiliaires presque en place
    \item Minimisation de la mémoire utilisée
  \end{itemize}
\end{frame}

\begin{frame}
  \frametitle{\insertsubsection}
  \framesubtitle{Optimisation importante}
  \begin{itemize}
    \item
    Terminer avec une multiplication naive plus tôt dans la récursion
    \item Taille optimale trouvée $\approx 50$
    \item Amélioration remarquable
  \end{itemize}
\end{frame}

\subsection{Complexité}
\begin{frame}
  \frametitle{\insertsubsection}
  \begin{itemize}
    \item Temps :
    \begin{itemize}
      \item $O(n^{\log_2 7})\approx O(n^{2.81})$
      pour deux matrices carrées
      \item $O(\frac{m\cdot n\cdot p}{min(m,n,p)^{3-\log_2 7}})\approx
      O(\frac{m\cdot n\cdot p}{min(m,n,p)^{0.19}})$
      en général
    \end{itemize}
    \item Espace :
    \begin{itemize}
      \item $O(m\cdot n+n\cdot p+m\cdot p)$ en plus des matrices en entrée ;
      transparent dans le code
    \end{itemize}
  \end{itemize}
\end{frame}

\subsection{Performance}
\begin{frame}
  \frametitle{\insertsubsection}
  \begin{tikzpicture}
  \begin{axis}[
    xlabel=n,
    ylabel=time (s)
  ]
  \addplot table[x=n,y=naive] {datastrassen.dat};
  \addplot table[x=n,y=strssn1] {datastrassen.dat};
  \addplot table[x=n,y=strssn2] {datastrassen.dat};
  \end{axis}
  \end{tikzpicture}
\end{frame}


\section{Parenthésage optimal}
\subsection{Principe}
\begin{frame}
  \frametitle{\insertsubsection}
  \begin{itemize}
  \item Produit de $n$ matrices
  % C'est bon là, pas la peine de réinventer la roue
  % Tu dis juste que le produit de matrices est associatif
  % non-commutatif en général. Il suffit donc de trouver le
  % parenth. opti.
  \item Différents coûts
  \item Programmation dynamique
  \end{itemize}
%%%%%%%%%%%%%%%%%%%%%%%%%%%%%%%%%%%%%%%%%%%%%%%%%%%%%%%%%%
%%%%%%%%%%%%%%%%%%%%%%%%MEC%%%%%%%%%%%%%%%%%%%%%%%%%%%%%%%
%%% en bas ca va pas dans le transparent on est d'accord ?

  % Un produit de $n$ matrices est défini en se ramenant à $n - 1$ produit de
  % $2$ matrices. Par associativité, on peut effectuer ces produits dans
  % n'importe quel ordre, tant que l'on conserve l'ordre des matrices
  % (c'est-à-dire qu'on essaye pas de les faire commuter).

  % En revanche, le nombre d'opérations effectuées pour les différents
  % parenthésages possibles n'est généralement pas le même. Par exemple :

  % \[\mbox{Soit } A_1, A_2, A_3 \mbox{ des matrices de tailles respectives }
  % 100\times 2, 2 \times 8000, 8000 \times 4\]

\end{frame}

\begin{frame}
  \frametitle{\insertsubsection}
  \framesubtitle{Exemple}
  Produit $A_1 \times A_2 \times A_3$
  \[
  \begin{array}{cc} 
    \mbox{Matrice} & \mbox{Taille}\\
    A_1 & 1000 \times 2\\
    A_2 & 2 \times 8000\\
    A_3 & 8000 \times 400
  \end{array}
  \]
  donne :
  \[
  \begin{array}{cc}
    \mbox{Parenthésage} & \mbox{Coût (algorithme naïf)}\\
    (A_1 \times A_2) \times A_3 & 3.216 \times 10^9\\
    A_1 \times (A_2 \times A_3) & 7.2 \times 10^6
  \end{array}
  \]
\end{frame}

\subsection{Structure du problème}
\begin{frame}
  \frametitle{\insertsubsection}
  \begin{itemize}
    \item Détermination récursive des «~coupures~»
    \item Exemple : 
      \[
      \begin{array}{lcccccccr}
        \Big((A_1 & \cdot & A_2) & \cdot & (A_3 & \cdot & A_4)\Big) & \cdot & A_5\\
        & \uparrow & & \uparrow & & \uparrow & & \uparrow \\
        & 3 & & 2 & & 3' & & 1\\
      \end{array}
      \]
    \item Respecte la propriété de sous-structure optimale
    \item Programmation dynamique
  \end{itemize}
\end{frame}

\subsection{Structure de données}
\begin{frame}
  \frametitle{\insertsubsection}
  \begin{itemize}
    \item Arbre de coupures
    \item Représentation compacte
  \end{itemize}
\end{frame}

\begin{frame}
  \frametitle{\insertsubsection}
  \framesubtitle{Exemple}
  \begin{itemize}
    \item $((A_1 \cdot A_2) \cdot (A_3 \cdot A_4)) \cdot A_5$
    \item Arbre de coupures :\\
       \Tree[.4
              [.2
                [.1 $A_1$ $A_2$ ]
                [.1 $A_3$ $A_4$ ]]
              $A_5$ ]\vspace{15pt}
    \item Parcours préfixe : 
      $\begin{array}{|c|c|c|c|}\hline 4 & 2 & 1 & 1\\\hline\end{array}$
  \end{itemize}
\end{frame}

\subsection{Complexité}
\begin{frame}
  \frametitle{\insertsubsection}
  Pour une chaîne de $n$ matrices :
  \begin{itemize}
    \item Temps : $O(n^3)$
    \item Espace : $O(n^2)$
  \end{itemize}
\end{frame}


\begin{frame}{Conclusion}
  \begin{itemize}
    \item Des algorithmes de complexités asymptotiques meilleures
    sont connus pour le parenthésage et la multiplication.
    \item Algorithme de Strassen plus efficace dans les
    applications pratiques (pour l'instant)
  \end{itemize}
\end{frame}

\end{document}
