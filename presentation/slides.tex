\documentclass{beamer}
\usepackage[french]{babel}
\usepackage[utf8]{inputenc}
%\usetheme{Warsaw}
\title[Projet de programmation et d'algorithmique]{
Algorithme de Strassen et parenthésage optimal
pour la mutliplication de matrices}
\author{Antoine Voizard, Li-yao Xia}
\date{18 décembre 2012}


\begin{document}

\begin{frame}
\titlepage
\end{frame}

\AtBeginSection[]
{
  \begin{frame}<beamer>
    \frametitle{\insertsection}

    \tableofcontents[currentsection]
  \end{frame}
}

\begin{frame}{Introduction}
  Objectifs :
  \begin{itemize}
    \item Ecrire du C.
    \item Faire mieux que les algorithmes naifs.
  \end{itemize}
\end{frame}

\section{Algorithme de Strassen}

\subsection{Principe}
\begin{frame}
  \frametitle{\insertsubsection}
  \begin{itemize}
    \item Multiplication rapide de matrices sur un anneau ;
    (ici, des {\bf int})
    \item Limiter le nombre de multiplications, en taille $2 \times 2$,
    \begin{itemize}
      \item mult. naïve : 8 multiplications scalaires,
      \item mult. de Strassen : 7 multiplications ;
    \end{itemize}
    \item Calcul par blocs, algorithme récursif ;
  \end{itemize}
\end{frame}

\subsection{Implémentation et optimisation}
\begin{frame}
  \frametitle{\insertsubsection}
  \framesubtitle{Structures de données}
  \begin{itemize}
    \item Matrice $A$ de taille $m\times n$
    \begin{itemize}
      \item Tableau {\bf A} d'{\bf int}
      (pointeur sur la première case) de taille {\bf m*n},
      %alloué dynamiquement
      \item Accès à la case $(i,j)$ : {\bf A[i*n+j]} ;
    \end{itemize}
    \item<2-> Sous-matrice $M$ extraite de $A$ de taille $m'\times n'$
    \begin{itemize}
      \item Pointeur sur la première case de $M$ dans {\bf A},
      \item Largeur de la matrice $A$, {\bf n},
      \item Accès à la case $(i,j)$ : {\bf M[i*n+j]}.
    \end{itemize}
  \end{itemize}
\end{frame}

\begin{frame}
  \frametitle{\insertsubsection}
  \framesubtitle{Addition et soustraction de sous-matrices}
\end{frame}

\subsection{Complexité}
\begin{frame}
  \frametitle{\insertsubsection}
\end{frame}

% Instabilité numérique


\section{Parenthésage optimal}
\subsection{Principe}
\begin{frame}
  \frametitle{\insertsubsection}
\end{frame}

\subsection{Implémentation}
\begin{frame}
  \frametitle{\insertsubsection}
\end{frame}
%représentation des données

\subsection{Complexité}
\begin{frame}
  \frametitle{\insertsubsection}
\end{frame}
% A faire :
% Complexité, espace, temps


%%%%%
%%%%% Derniers slides

\begin{frame}{Conclusion}
  Des algorithmes de complexités asymptotiques meilleures
  sont connus pour le parenthésage et la multiplication.

  Mais l'algorithme de Strassen est plus efficace dans les
  applications pratiques. (pour l'instant)
\end{frame}

\end{document}
