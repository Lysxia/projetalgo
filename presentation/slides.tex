\documentclass{beamer}
\usepackage[french]{babel}
\usepackage[utf8]{inputenc}
\usepackage{pgfplots}
\usepackage{listings}
%\usetheme{Warsaw}
\title[Projet de programmation et d'algorithmique]{
Algorithme de Strassen et parenthésage optimal
pour la mutliplication de matrices}
\author{Antoine Voizard, Li-yao Xia}
\date{18 décembre 2012}


\begin{document}

\begin{frame}
\titlepage
\end{frame}

\AtBeginSection[]
{
  \begin{frame}<beamer>
    \frametitle{\insertsection}

    \tableofcontents[currentsection]
  \end{frame}
}

\begin{frame}{Introduction}
  Objectifs :
  \begin{itemize}
    \item Ecrire du C ;
    \item Faire mieux que les algorithmes naifs.
  \end{itemize}
\end{frame}

\section{Algorithme de Strassen}

\subsection{Principe}
\begin{frame}[fragile]
  \frametitle{\insertsubsection}
  \begin{itemize}
    \item Multiplication rapide de matrices sur un anneau ;
    (ici, des \verb=int=)
    \item Limiter le nombre de multiplications, en taille $2 \times 2$,
    \begin{itemize}
      \item mult. naïve : 8 multiplications scalaires,
      \item mult. de Strassen : 7 multiplications ;
    \end{itemize}
    \item Calcul par blocs, algorithme récursif.
  \end{itemize}
\end{frame}

\subsection{Implémentation et optimisation}
\begin{frame}[fragile]
  \frametitle{\insertsubsection}
  \framesubtitle{Structures de données}
  Représentation compacte des données.
  \begin{itemize}
    \item<2-> Matrice $A$ de taille $m\times n$
    \begin{itemize}
      \item Tableau \verb=A= d'\verb=int=
      (pointeur sur la première case) de taille \verb=m*n=,
      %alloué dynamiquement
      \item Accès à la case $(i,j)$ : \verb=A[i*n+j]= ;
    \end{itemize}
    \item<3-> Sous-matrice $M$ extraite de $A$ de taille $m'\times n'$
    \begin{itemize}
      \item Pointeur sur la première case de $M$ dans \verb=A=,
      \item Largeur de la matrice $A$, \verb=n=,
      \item Accès à la case $(i,j)$ : \verb=M[i*n+j]=.
    \end{itemize}
  \end{itemize}
\end{frame}

\begin{frame}[fragile]
  \frametitle{\insertsubsection}
  \framesubtitle{Implémentation}
  \small\begin{verbatim}
  int* strassen(int* A, int* B, int m, int n, int p);
  \end{verbatim}\normalsize

  \begin{itemize}
    \item Opérations auxiliaires presque en place.
    \item Minimisation de la mémoire utilisée.
  \end{itemize}
\end{frame}

\begin{frame}
  \frametitle{\insertsubsection}
  \framesubtitle{Optimisation importante}
  \begin{itemize}
  \item
  Terminer avec une multiplication naive plus tôt dans la récursion ;
  \item Taille optimale trouvée $\approx 50$.
  \end{itemize}
\end{frame}

\subsection{Complexité}
\begin{frame}
  \frametitle{\insertsubsection}
  \begin{itemize}
    \item Temps :
    \begin{itemize}
      \item $O(n^{\log_2 7})\approx O(n^{2.81})$
      pour deux matrices carrées,
      \item $O(\frac{m\cdot n\cdot p}{min(m,n,p)^{3-\log_2 7}})\approx
      O(\frac{m\cdot n\cdot p}{min(m,n,p)^{0.19}})$
      en général ;
    \end{itemize}
    \item Espace :
    \begin{itemize}
      \item $O(m\cdot n+n\cdot p+m\cdot p)$ en plus des matrices en entrée ;
      transparent dans le code.
    \end{itemize}
  \end{itemize}
\end{frame}

\subsection{Performance}
\begin{frame}
  \frametitle{\insertsubsection}
  \begin{tikzpicture}
  \begin{axis}[
    xlabel=n,
    ylabel=time (s)
  ]
  \addplot table[x=n,y=naive] {datastrassen.dat};
  \addplot table[x=n,y=strssn1] {datastrassen.dat};
  \addplot table[x=n,y=strssn2] {datastrassen.dat};
  \end{axis}
  \end{tikzpicture}
\end{frame}


\section{Parenthésage optimal}
\subsection{Principe}
\begin{frame}
  \frametitle{\insertsubsection}
\end{frame}

\subsection{Implémentation}
\begin{frame}
  \frametitle{\insertsubsection}
\end{frame}
%représentation des données

\subsection{Complexité}
\begin{frame}
  \frametitle{\insertsubsection}
\end{frame}
% A faire :
% Complexité, espace, temps


%%%%%
%%%%% Derniers slides

\begin{frame}{Conclusion}
  Des algorithmes de complexités asymptotiques meilleures
  sont connus pour le parenthésage et la multiplication.

  Mais l'algorithme de Strassen est plus efficace dans les
  applications pratiques. (pour l'instant)
\end{frame}

\end{document}
